\documentclass[openany]{book}

\usepackage{inctech}
\graphicspath{ {./images/} }

%temp


\begin{document}

% Front matter
\frontmatter
\tableofcontents
\clearpage

% Main content
\mainmatter

\raggedright

\part{Chemistry - A Molecular Approach}

\chapter{Matter, Measurement, and Problem Solving}

\section{null}

\section{The Scientific Approach to Knowledge}

Steps of scientific approach:
\begin{itemize}
	\item Observation: gathering of data (qualitative \& quantitative)
	\item Hypothesis: Tentative explanation for the observation
	\item Scientific Law: Tentative explanation for the past observation and predicts future ones
	\item Experiments:
	      \begin{itemize}
		      \item Performed to test the validity of the hypothesis
		      \item Always produce new information
	      \end{itemize}
	\item Scientific theory: They explain why an observation occurs
\end{itemize}

\includegraphics[width=\textwidth]{scientitfic-approach-diagram.png}

\section{The Classification of Matter}

Matter: anything that has mass and occupies space

Substance: A specific instance of matter

Matter is classfied according to its state and composition

\subsection{The States of Matter}

Matter exists in three different states: solid, liquid,
and gas

\begin{tabular}{c | c c c}
	              & Solid & Liquid & Gas        \\
	\midrule
	Volume        & 
	Constant      & 
	Constant      & 
	Changing                                    \\
	Shape         & 
	Constant      & 
	Changing      & 
	Changing                                    \\
	Density       & 
	High          & 
	Medium        & 
	Low                                         \\
	Structure     & 
	\makecell{Packed closely at fixed locations \\ They may be crystalline or amorphous}&
	\\
	Movement      & 
	Vibrate       & 
	relative flow & 
	\\
	Compression   & 
	No            & 
	No            & 
	Yes                                         \\
	
\end{tabular}

\section{misc}

The 6 types of phase changes:
\begin{enumerate}
	\item Diposition
	\item Sublimation
	\item Melting
	\item Freezing
	\item Vaporization
	\item Condensation
\end{enumerate}

The melting and freezing of all pure substances are the same
The vaporization and condensation of pure substances are the same.

\subsection{Laws:}

Antoine Lavoisier

The Law of Conservation of mass

Matter is neither nor destroyed in a chemical reaction

Joseph Proust

The Law of Definite Proportions or Constant Composition

All samples of a given compound always have the same proportions of their constituent elemnts

John Dalton

The Law of Multiple Proportions

When 2 masses combine to form more than 1 compound, a fixed mass of 1 element combine with the a fixed mass of the other element in small whole number ratios

v

\section{Units \& Measurements}

Units: 
—standard quantities used to specify
measurements


Types of Unit Systems:

Two most common unit systems:
Metric system, used in most of the world
English system, used in the United States

Scientists use the International System of Units (SI),
which is based on the metric system

Some Units:

The temperature is the average amount of kinetic energy of the atoms and molecules that compose the matter.

The Fahrenheit degree is approximately five-ninths of a a Celsius degree

\chapter{Periodic Properties of Elements}

\chapter{Molecules and Compounds}



\section{Compounds}

Most ionic compundds dissolve in water

Most covalent bonds don't dissolve in water

\section{Chemical Bonds}

\subsection{Ionic Bonds}

\begin{itemize}
	\item Ionic bonds involve the complete transfer of electrons from the metal atom to the nonmetal atom
	\item The metal atom becomes a cation while the nonmetal atom becomes an anion
	\item These oppositely charged ions attract one another by electrostatic forces and form an ionic bond
	\item It's wrong to call ionic compounds molecules as they exist as crystals
\end{itemize}

\subsection{Covalent Bonds}

\begin{itemize}
	\item Covalent bonds occur between two or more nonmetals
	\item The two atoms share electrons between them, forming a molecule
	\item Covalently bonded compounds are also called molecular compounds
\end{itemize}

Atomic elements:

All noble gasses and metals can exist as atomic elements

Only 2 positive polyatomic ions:
\[
	NH_4^+
\]
\[Hg_2^+\]

\chapter{Molecules and Compounds}

Filling of electrons:
d subshell has more energy in s subshell

Taking electrons:
we take electrons from the highest floor.

\section{Acids}

\begin{itemize}
	\item Sour taste
	\item Dissolve many metals:
	      such as Zn, Fe, and Mg; but not Au, Ag, or Pt
	      
	      When you mix metals and acids, H\textsubscript{2} bubbles get released
	\item Acids are composed of hydrogen, usually written first in their formulas, and one or more nonmetals, written second
	\item Acids don't react with other acids
\end{itemize}

\subsection{Binary Acids}

Binary acids do not contain oxygen

HCN and H\textsubscript{2}S are binary acids because they do not contain oxygen.

There are only 6 Binary acids

\subsubsection{Nomenclature}

• Write a hydro- prefix.
• Follow with the nonmetal base name.
• Add -ic.
• Write the word modified polyatomic ion name and acid at the beginning and end, respectively, of the name.

\subsection{Oxyacids}

Oxygen is present in the compound

\subsubsection{Nomenclature}

\begin{itemize}
	\item If the polyatomic ion name ends in -ate, change ending to -ic.
	\item If the polyatomic ion name ends in -ite, change ending to -ouIf the polyatomic ion name ends in -ate, change ending to -ic.
	      If the polyatomic ion name ends in -ite, change ending to -ous.
	      Write word acid at the end of all names.
	      s.
	\item Write word acid at the end of all names.
\end{itemize}

\chapter{Names of compunds}

\begin{tabular}{L c}
	IO_2        & Iodite       \\
	S_2O_3^{2-} & Borite ion   \\
	S_2O_3^{2-} & Thiosulphite \\
\end{tabular}

\section{Formula Mass}

The mass of an individual molecule or formula unit also known as molecular mass (MM)

Sum of the masses of the atoms in a single molecule or formula unit

whole = sum of the parts!

in terms of magnitude (excluding units):

Molar mass = Formula mass,

Molar mass = Atomic mass

Conversion Factors from Chemical Formulas:

Chemical formulas show the relationship between numbers of atoms
and molecules.

Or moles of atoms and molecules

These relationships can be used to determine the amounts of
constituent elements and molecules.
Like percent composition


\chapter{Basic Concepts of Chemical Bonding}

Chemical reactions are exothermic because they lose energy to conduct the reaction

\section{Lewis structure of Atoms}

It doesn't matter which in which side you begin to place dots for electrons. You can begin from the right, left, up or down. Even for pairing.

\chapter{The Quantum Mechanical Model of The Atom}\label{chap:The Quantum Mechanical Model of The Atom} % (fold)

\begin{outline}
	\1 \textbf{Principal Quantum Number} ($n$):  
	\2 Represents the main energy level (shell).
	\2 Takes natural number values: $n = 1, 2, 3, \dots$
	\2 Determines the maximum number of possible subshells.
	
	\1 \textbf{Azimuthal (Orbital Angular Momentum) Quantum Number} ($l$):  
	\2 Defines the shape of the orbital (subshell).
	\2 Depends on $n$: $l$ can take integer values from $0$ to $(n-1)$.
	\2 Subshell notation:  
	\3 $l = 0$ (s-orbital)
	\3 $l = 1$ (p-orbital)
	\3 $l = 2$ (d-orbital)
	\3 $l = 3$ (f-orbital)
	
	\1 \textbf{Magnetic Quantum Number} ($m_l$):  
	\2 Specifies the orientation of the orbital in space.
	\2 Depends on $l$: $m_l$ can take integer values from $-l$ to $+l$.  
	\2 Example: If $l = 2$, then $m_l = -2, -1, 0, 1, 2$.
	
	\1 \textbf{Spin Quantum Number} ($m_s$):  
	\2 Represents the intrinsic spin of an electron.
	\2 Independent of other quantum numbers.
	\2 Can take only two possible values: $m_s = +\frac{1}{2}, -\frac{1}{2}$.
\end{outline}

% chapter The Quantum Mechanical Model of The Atom (end)

\end{document}
