\documentclass[openany]{book}

\usepackage{inctech}
\graphicspath{ {./images/} }

%temp

\begin{document}

% Front matter
\frontmatter
\tableofcontents
\clearpage

% Main content
\mainmatter

\raggedright

\section{Significant digits}

\begin{itemize}
    \item All non zeros are significant
    \item Interior Zeros are not significant
    \item Leading zeroes are not significant
    \item Trailing zeros:
        \begin{itemize}
            \item with decimal point are significant
            \item without decimal point are not significant
            \item before an implied decimal point is ambiguous and should be avaided
        \end{itemize}
\end{itemize}

\subsection{Exact numbers}

They have an inifinite number of significant figures ( no uncertainty )

They can be 2 things:

\begin{itemize}
    \item Countable number
    \item From a definition    
\end{itemize}

\subsection{Addition \& Subtraction}

The result should have the same number of precision digits as the number with the least number of precision digits. 

ex: \( 132.853-5=128 \)

\subsection{Multiplication \& Division}

    The result should have the same number of significant digits as the number with the least S.D.

\section{Notations}

\begin{itemize}
    \item Floating
    \item Scientific notation

    \(A.B.. \times 10^n \), where A is b/w 1-9
    \item Engineering notation
    
    \( AAA.BB.. \times 10^n \), where n is a multiple of 3 \& only 3 digits are before the decimal point
\end{itemize}

\section{Math Prefixes \& Engineering Notation}

Uppercases letter are used for abbreviations of the prefixes of +ve powers ivolving 10, while lowercase is used for -ve powers 

\chapter{}

\begin{tabular}{c c c}
    Conductor & Semi-conductor & Insulator \\
    \hline
    Loosely bound e&
        more tightly bound e&
        tightly bound e\\
    
    Best :
    with 1 electron in outer shell &
    Usually 4 e in outer shell&
    with 6-8 electron in outer shell\\
    \\
\end{tabular}

\subsection{Power}

\(P=VI\)

Electrical Horsepower

\(1\ hp \cong 746\ watts \)

\chapter{Parallel Circuits}

\begin{equation*}
    R_T = \Sigma \frac{1}{R_n}
\end{equation*}

In a parallel circuit, the total resistance should be less than the resistor with lowest resistance. 

\section{Kirchoff's Current Law (KCL)}

\begin{align*}
    \Sigma I_i &= \Sigma I_o \\
    \text{or} \\
    \Sigma I &= 0 \\
\end{align*}

\section{Current Division in Parallel Circuits}

The voltage V can then be determined using Ohm’s law

Since the voltage V is the same across parallel elements.

The current divider rule:
\[I_x=\frac{R_T}{R_x}I_T\]

Proof:



For 2 resistors:

\begin{align*}
    \text{For I\textsubscript{1}:} \\ I_1 = \frac{I_TR_2}{R_1+R_2} \\
    \text{For I\textsubscript{2}:} \\ I_2 = \frac{I_TR_1}{R_1+R_2} \\
\end{align*}

Proof:

\chapter{Series-Parallel Circuits}

Current source can't be connected in series. Only in parallel. 

\end{document}
