\documentclass{book}

\usepackage{inctech}

\begin{document}

\section{Unit 3: Exponential and Logarithmic
  Functions}\label{unit-3-exponential-and-logarithmic-functions}

\subsection{Exponential Functions}\label{exponential-functions}

Let \(b\) be a constant real number such that \(b>0\) and \(b\neq1\)

Then for any real number \(x\),a function of the form \(f(x)=b^{x}\) is called an exponential function of base \(b\)


\begin{table}[ht]
    \centering
    \begin{tabular}{L L}
        \toprule
        \text{Exponential Functions} & \text{Not Exponential Functions} \\
        \midrule
        f(x) = 3^x & m(x) = x^2 \rightarrow \text{ base is not constant}  \\
        g(x) = \left(\frac{1}{3}\right)^x & n(x) = \left(-\frac{1}{3}\right)^x \rightarrow \text{ base is negative} \\
        h(x) = (\sqrt{2})^x & p(x) = 1^x \rightarrow \text{ base is 1} \\
        \bottomrule
    \end{tabular}
    \caption{Comparison of Exponential and Non-Exponential Functions}
\end{table}



Formulas from the Law of Exponents

\begin{align*}
	a^{-n}&=\dfrac{1}{a^{n}} \\
	a^{n}&=\dfrac{1}{a^{-n}} \\
	\left(\dfrac{a}{b}\right)^{-n}&=\left(\dfrac{b}{a}\right)^{n} \\
\end{align*}

\section{Graph of \(f(x)=b^x\)}

The graph of an exponential function defined by
\(f(x)=b^{x} \text{, where }(b>0 \text{ and } b \neq 1)\) has the following properties.

\begin{enumerate}
	\item
	      If \(b>1,f\) is an increasing exponential function, sometimes called an exponential growth function.
		  
		  If \(0<b<1,f\) is a decreasing exponential function, sometimes called an exponential decay function.
	\item
	      Domain = \(\{\mathbb{R}\}\)
	\item
	      Range = \((0,\infty)\)
	\item
	      The line \(y=0\) (x-axis) is a horizontal asymptote.
	\item
	      The function passes through the point (0,1) because \(f(0)=b^{0}=1\)
\end{enumerate}

\includegraphics{./images/fGP3mu6QKSVgVLH6nREnw4P1ITM8nFdlb.png}

\textbf{Vertical and Horizontal Translations on Exponential
	Functions}\label{vertical-and-horizontal-translations-on-exponential-functions}

\[f(x) = ab^{x-h}+k\]

If:

\begin{itemize}
	\item
		\(h>0\), shift to the right
		
		$h<0$, shift to the left
	\item
	      \(k>0\), shift upward.

	      \(k<0\) shift downward.
	\item
	      \(a<0\), reflect across the \(x\) -axis.

	      \(0<|a|<1\), shrink vertically

	      \(|a|>1\), stretch vertically
\end{itemize}

\section{Exponential Function base e}\label{exponential-function-base-e}

The value of the number \(e\) is irrational (non-terminating,
non-repeating decimal) just like \(\pi\). It is a universal constant.

\(e \approx 2.718281828459 \dots\)

Because the base \(e\) is greater that 1,the graph of \(f(x)=e^{x}\) is
an increasing function.

\begin{tabular}{@{}cc@{}}
	\includegraphics[width=0.45\textwidth]{./images/fHHVgtHUP6EgzQDPfdlPngaLqZgl3fmOL.png} &
	\includegraphics[width=0.45\textwidth]{./images/fbegsE1gcNnr36GGTLSpWPXuyZzrgWkSy.png} \\
\end{tabular}

\section{Logarithmic Functions}\label{logarithmic-functions}

\emph{What are Logarithmic Functions?}

They are the inverses of exponential functions.

If \(x\) and \(b\) are positive real numbers such that \(b\neq1\) ,then
\(y=\log_{b}x\) is called the logarithmic function of base \(b\)

\includegraphics[width=\dimexpr\textwidth-0.5in\relax]{./images/f3TBlML5lhEUiowGpaAqM9xM9EWwX4Fxb.png}

The logarithmic function base \(b\) is defined as the inverse of the
exponential function base \(b\)

\includegraphics[width=\dimexpr\textwidth-0.5in\relax]{./images/fnPFV9UAK0p7W3qpF8SPxKg3rBvZEigWo.png}

\begin{table}[ht]
    \centering
    \begin{tabular}{>{\raggedright}p{2cm} L p{7cm}}  % Adjusting column widths
        \toprule
        Concept & Equation & Explanation \\
        \midrule
        Exponential function & f(x) = b^x & First, replace \( f(x) \) by \( y \). \\
        & y = b^x & Next, interchange \( x \) and \( y \). \\
        Inverse of exponential function & x = b^y & This equation provides an implicit relationship between \( x \) and \( y \). To solve for \( y \) explicitly (that is, to isolate \( y \)), we must use logarithmic notation. \\
        Logarithmic function & y = \log_b x & \\
        \bottomrule
    \end{tabular}
    \caption{Relationship between Exponential and Logarithmic Functions}
\end{table}

\subsubsection{Logarithmic Vocabulary}\label{logarithmic-vocabulary}

\includegraphics[width=\dimexpr\textwidth-0.5in\relax]{image-4.png}


\subsubsection{Basic Properties for
	Logarithms}\label{basic-properties-for-logarithms}

\includegraphics[width=\dimexpr\textwidth-0.5in\relax]{./images/frPQwcBUQQUktTPuSCMNCZCb3u1fqV7H7.png}

\subsubsection{Graph Logarithmic Functions}

Since a logarithmic function \(y=\log_{b}x\) is the inverse of the
exponential function \(y=b^{x}\), their graphs must be symmetric with
respect to the line \(y=x\)

\includegraphics[width=\dimexpr\textwidth-0.5in\relax]{./images/fRnL8N0N2G9Qbdic8dam7NGvt9Gc9uDBP.png}

\begin{tabular}{@{}p{0.45\textwidth} p{0.45\textwidth}@{}}
	\toprule
	Exponential Functions & Logarithmic Functions \\
	\midrule
	\includegraphics[width=\linewidth]{./images/fS4hiBVwClXaP4WMhWFGbxGfk5OWaZChf.png} & 
	\includegraphics[width=\linewidth]{./images/fKkrD4DS6U4OsdXllNUZu4wMh5OLh2GSE.png} \\
	Domain: \((0,\infty)\) & Domain: -cco \\
	Range: & \\
	 & Vertical asymptote: \(x=0\) \\
	\bottomrule
\end{tabular}


Passes through (1,0) If \(b>1\) ,the function is increasing. If
\(0<b<1\) the function is decreasing

Range: \((0,\infty)\) Horizontal asymptote: \(y=0\) Passes through (0,
1) If \(b>1\) ,the function is increasing. If \(0<b<1\) , the function
is decreasing,

\subsection{Using Transformations to Graph Logarithmic
	Functions}\label{using-transformations-to-graph-logarithmic-functions}

12.Graph the function \(f(x)=\log_{2}(x+3)-2\) Identify the vertical
asymptote and write the. domain in interval notation..

\begin{itemize}
	\tightlist
	\item
	      table \(2^x\)
\end{itemize}

\begin{longtable}[]{@{}ll@{}}
	\toprule\noalign{}
	x  & y   \\
	\midrule\noalign{}
	\endhead
	\bottomrule\noalign{}
	\endlastfoot
	-2 & 1/4 \\
	-1 & 1/2 \\
	0  & 1   \\
	1  & 2   \\
	2  & 4   \\
	3  & 8   \\
\end{longtable}

\begin{itemize}
	\tightlist
	\item
	      Table \(\log_2x\)
\end{itemize}

\begin{longtable}[]{@{}ll@{}}
	\toprule\noalign{}
	x   & y  \\
	\midrule\noalign{}
	\endhead
	\bottomrule\noalign{}
	\endlastfoot
	1/4 & -2 \\
	1/2 & -1 \\
	1   & 0  \\
	2   & 1  \\
	4   & 2  \\
	8   & 3  \\
\end{longtable}

\begin{itemize}
	\tightlist
	\item
	      Transformations

	      \begin{itemize}
		      \tightlist
		      \item
		            left 4
		      \item
		            down 2
	      \end{itemize}
\end{itemize}

\includegraphics[width=\dimexpr\textwidth-0.5in\relax]{./images/fKGfRD9wyOU5DFva4X9ouvSdQ63ORGotC.png}

13(HW). Graph the function \(f(x)=\log_{3}(x-4)+1\) . Identify the
vertical asymptote and write the. domain in interval notation

\includegraphics[width=\dimexpr\textwidth-0.5in\relax]{./images/feTNngQiFhtxiQ68PbpNA2LdvgyBgOTiw.png}

\includegraphics[width=\dimexpr\textwidth-0.5in\relax]{./images/f5GnIS2DrKYKoCTqtO5uVCA1GLuHy1E9P.png}

Domain of a Logarithmic Function.

\begin{enumerate}
	\def\labelenumi{\arabic{enumi}.}
	\setcounter{enumi}{13}
	\tightlist
	\item
	      Write the domain of the following functions in interval notation and
	      identify the vertical asymptote(s)
\end{enumerate}

\begin{enumerate}
	\def\labelenumi{\alph{enumi}.}
	\tightlist
	\item
	      \(p(x)=\log_{2}\left(2x+4\right)\)
\end{enumerate}

\begin{align}
		2x+4 > 0 \\ 2x > -4 \\ x > -2 \\
	\end{align}

\begin{enumerate}
	\def\labelenumi{\alph{enumi}.}
	\setcounter{enumi}{1}
	\tightlist
	\item
	      \(g(x)=\ln{(5-x)}\)
\end{enumerate}

\textbf{A} \begin{align}
		5-x > 0 \\ x > 5 \\
	\end{align}

\begin{enumerate}
	\def\labelenumi{\alph{enumi}.}
	\setcounter{enumi}{2}
	\tightlist
	\item
\end{enumerate}

\textbf{A}

\begin{itemize}
	\tightlist
	\item
	      \begin{align}
			      x^2-9 >0 \\ (x-3)(x+3) > 0 \\
		      \end{align}
	\item
	      draw number line and do checks
	\item
	      VA: x=3 \& x=-3
	\item
	      Domain: \((-\infty,-3)\cup(-3,\infty)\)
\end{itemize}

\includegraphics[width=\dimexpr\textwidth-0.5in\relax]{./images/fXgts423wq5PkLGqcvKEwtem2dETaI8tW.png}

\includegraphics[width=\dimexpr\textwidth-0.5in\relax]{./images/f4gCEbKU3r3TwCByHSVdrpM9nkxHEnVIm.png}

\includegraphics[width=\dimexpr\textwidth-0.5in\relax]{./images/fSC9EQDxXlqxI1YgfQQ3kFgrOYKy9efOg.png}

\includegraphics[width=\dimexpr\textwidth-0.5in\relax]{./images/feMPYMBc6mPKDYcLgXVyBO5YXGgY0hgiy.png}

\includegraphics[width=\dimexpr\textwidth-0.5in\relax]{./images/fFhmo8CsubHRRGlVAiFzkg78XzEisOIAo.png}

\includegraphics[width=\dimexpr\textwidth-0.5in\relax]{./images/fDLClMTVMDGzGZoKBFW1qVhaHu4g8kGTz.png}

\includegraphics[width=\dimexpr\textwidth-0.5in\relax]{./images/fkZSWk2zI3nYsY0r3qTRhIzFxdZG1w836.png}

\(\frac{\frac{\sqrt{0}}{\sqrt{0}}}{\frac{\sqrt{0}}{\sqrt{0}}}\)

\subsubsection{Product, Quotient and Power
	Properties.}\label{product-quotient-and-power-properties.}

Because exponential functions satisfy properties like:

\begin{longtable}[]{@{}lll@{}}
	\toprule\noalign{}
	\((a^m)^n=a^{m \cdot n}\) & \(a^m \cdot a^n\) &
	\(\frac{a^m}{a^n}=a^{m-n}\)                     \\
	\midrule\noalign{}
	\endhead
	\bottomrule\noalign{}
	\endlastfoot
	(x\^{}2)                  &                   & \\
\end{longtable}

\includegraphics[width=\dimexpr\textwidth-0.5in\relax]{./images/fmBIwgygRumPEa3cc7Iess7C7w9hlSWiF.png}

Logarithms satisfy these properties:

Product Property

\(\log_b(xy)=\) \(\log_{b}x+\log_{b}y\)

Terms must

Quotient Property

\(\log_{b}\) \(\stackrel{'x}{-}\) \(\log_bx-\) \(\log_{by}\)

have the sam base to be able to combine

Power Property

\(p(vr)_{1-x}\) 00

\(them.\)

Writing a Logarithmic Function in Expanded Form

15.Expand the expressions using Properties of Logs.

\begin{itemize}
	\tightlist
	\item
	      \(\log_2(8x)\)
\end{itemize}

\(\log_28+\log_2x\)

\begin{itemize}
	\tightlist
	\item
	      \(\log (\frac{x}{1000})\)
\end{itemize}


	\begin{align}
		= \log x - \log 1000 \\ \\
	\end{align}


\begin{enumerate}
	\def\labelenumi{\alph{enumi}.}
	\setcounter{enumi}{3}
	\tightlist
	\item
\end{enumerate}


	\begin{align}
		 & = \log_5 \frac{(x^5)^\frac{1}{2}}{y} \\&= \log_5 \frac{x^\frac{5}{2}}{y} \\&= \frac{5}{2} \log_5x - \log_5 y \\
	\end{align}


\begin{enumerate}
	\def\labelenumi{\alph{enumi}.}
	\setcounter{enumi}{4}
	\tightlist
	\item
\end{enumerate}

do later

\begin{enumerate}
	\def\labelenumi{\alph{enumi}.}
	\setcounter{enumi}{5}
	\tightlist
	\item
\end{enumerate}


	\begin{align}
		 & = \\
	\end{align}


\includegraphics[width=\dimexpr\textwidth-0.5in\relax]{./images/fB8FM4VXk0Yw4rp36GmUspIzBfGYLaz3x.png}

\subsection{Writing a Logarithmic Expression as a Single
	Logarithm}\label{writing-a-logarithmic-expression-as-a-single-logarithm}

16.Combine the expressions using Properties of Logs a.
\(\widehat{x\ln y}+\widehat{2\ln(x+8)}\)

\begin{enumerate}
	\def\labelenumi{\alph{enumi}.}
	\item
	      
		      \begin{align}
			       & = \ln y^x + \ln (x+8)^2 \\&= \ln (y^x \cdot (x+8)^2) \\
		      \end{align}
	      
	\item
	      \(3 \log (5) + \log (16) - \log(20)\)
\end{enumerate}


	\begin{align}
		\log5^3 + \log (16) - \log(20)
	\end{align}


\begin{enumerate}
	\def\labelenumi{\alph{enumi}.}
	\setcounter{enumi}{2}
	\tightlist
	\item
	      \((x+5) \cdot \frac{(x+2)^\frac{1}{3}}{x^4}\)
\end{enumerate}


	\begin{align}
		 & = \log_2 \left ( (x+5) \cdot \frac{(x+2)^\frac{1}{3}}{x^4}) \right ) \\
	\end{align}


\includegraphics[width=\dimexpr\textwidth-0.5in\relax]{./images/fOn7tsGeyRi4sA9gMt8nbERyxiwu7zB0g.png}

\includegraphics[width=\dimexpr\textwidth-0.5in\relax]{./images/fMBQffuG0AcfzbqxoaZ4m7dcfqr8YALxK.png}

\includegraphics[width=\dimexpr\textwidth-0.5in\relax]{./images/fe6u2zF77gGpN4YkRKgpLa3PPeygPlzH4.png}

\subsection{Change of Base Formula A calculator can be used to
	approximate the value of a logarithm with base 10 or base
	e.}\label{change-of-base-formula-a-calculator-can-be-used-to-approximate-the-value-of-a-logarithm-with-base-10-or-base-e.}

However,to use a calculator for any other base,we must use the change
of base formula.

\section{Change-of-Base Formula)}\label{change-of-base-formula}

Let \(a\) andb be positive real numbers such that \(a\neq1\) and
\(b\neq1.\) Then for any positive real number .x

\(\log_bx=\frac{\log_ax}{\log_ab}\)

Note:The change-of-base formula converts algorithm of one base to a
ratio of logarithms of a different base.For the purpose of using a
calculator.we often apply the change-of base formula with base 10 or base
e

\includegraphics[width=\dimexpr\textwidth-0.5in\relax]{./images/fdiEBH7lnmvt7CmvQXXgBZtdOC4fygfYD.png}

17.Use the change of base formula to approximate \(\log_4153\) by using
base 10.Round to 4 decimal places.

18.Estimate \(\log_4153\) between two consecutive integers.

19.Use the change of base formula to approximate \(\log_623\) by using
base e.Round to 2 decimal places. \(\frac{\ln23}{\ln6}=\)

20.Use the change of base formula to approximate \(\log_3100\) by using
base 10. Round to 4 decimal places. \#

\subsubsection{Recall: Order of
	Operations}\label{recall-order-of-operations}

The sequence to follow when performing operations in a mathematical
equation

\includegraphics[width=\dimexpr\textwidth-0.5in\relax]{./images/fSZWDzT2sKqFQrHBd3ZIkC8O1M7fHb7pt.png}

Example: Simplify \(-2(7-5)^{2}+1\cdot3\)


	\begin{align}
		 & = -2 \\
	\end{align}


\subsection{Properties of Logs}\label{properties-of-logs}

\includegraphics[width=\linewidth]{images/log-prop.png}

\subsection{Solving Exponential
	Equations}\label{solving-exponential-equations}

Circle which equation below will I use the technique of ``solving an
exponential equation''?.

\includegraphics[width=\dimexpr\textwidth-0.5in\relax]{./images/fDGZw0GvPP3I0FWyiQy2XgcBvsdGNKtkS.png}

For the following problems below, answers must be exact (fractions/roots
okay). No decimals.

\begin{enumerate}
	\def\labelenumi{\arabic{enumi}.}
	\setcounter{enumi}{20}
	\tightlist
	\item
	      Solve \(3^{2x-6}=81\)
\end{enumerate}


	\begin{align}
		 & = 3^{2x-6}=3^4 \\ \\
	\end{align}


22.\(ext{Solve 25}^{4-t}=\left(\frac{1}{5}\right)^{3t+1}\)


	\begin{align}
		 & = 25^{4-t}=\left(\frac{1}{5}\right)^{3t+1} \\&= (5^2)^{4-t}=\left(\frac{1}{5}\right)^{3t+1} \\&= \left ({\frac{1}{5}}^{-2} \right )^{4-t}=\left(\frac{1}{5}\right)^{3t+1} \\&= -2(4-t) = 3t +1 \\
	\end{align}


\section{Steps to Solving Exponential Equations by Using
  Logarithms.}\label{steps-to-solving-exponential-equations-by-using-logarithms.}

\begin{enumerate}
	\def\labelenumi{\arabic{enumi}.}
	\tightlist
	\item
	      Isolate the exponential expression on one side of the equation. 2.Take
	      a logarithm of the same base on both sides of the equation. 3. Use the
	      power property of logarithms to ``bring down'' the exponent. 4. Solve
	      the resulting equation.
\end{enumerate}

23.Solve \(7^{x}=60\) . Give BOTH an exact answer and an approximate
answer rounded to two decimal places.


	\begin{align}
		7^{x}=60 \\&= \log 7^{x}=\log 60 \\&= x\log 7=\log 60 \\&= x=\frac{\log 7}{\log 60} \\
	\end{align}


\begin{enumerate}
	\def\labelenumi{\arabic{enumi}.}
	\setcounter{enumi}{23}
	\tightlist
	\item
	      Solve \(10^{5+2x}+820=49,600\) Give BOTH an exact answer and an
	      approximate answer rounded to two decimal places.
\end{enumerate}


	\begin{align}
		10^{5+2x}+820=49,600 \\ 10^{5+2x}=49,600-820 \\ 10^{5+2x}=48,780\\ \log 10^{5+2x}=\log 48,780 \\ (5+2x)\log 10=\log 48,780 \\ (5+2x) \cdot 1=\log 48,780  \\ 5+2x =\log 48,780 \\ 2x =\log (48,780) -5 \\ x = \frac{\log (48,780) -5}{2} \\
	\end{align}


25.Solve for \(t\) in the equation \(2000=18,000e^{-0.4t}\) .Give BOTH
an exact answer and an approximate answer rounded to two decimal places.


	\begin{align}
		2000=18,000e^{-0.4t} \\ \ln \frac{1}{9}= \ln e^{-0.4t} \\ \ln \frac{1}{9}= -0.4t \\ t = \frac{\ln \dfrac{1}{9}}{-0.4}\\
	\end{align}


\begin{enumerate}
	\def\labelenumi{\arabic{enumi}.}
	\setcounter{enumi}{25}
	\tightlist
	\item
	      Solve \(4^{2x-7}=5^{3x+1}\) . Give BOTH an exact answer and an
	      approximate answer.
\end{enumerate}


	\begin{align}
		4^{2x-7} & =5^{3x+1} \\ \log (4^{2x-7})&=\log (5^{3x+1}) \\ (2x-7) \log 4 &= (3x+1) \log 5 \\ 2x \log 4 - 7 \log 4 &= 3x\log 5 + \log 5 \\  \\
	\end{align}


\begin{enumerate}
	\def\labelenumi{\arabic{enumi}.}
	\setcounter{enumi}{26}
	\item
	      Solve \(\left(\frac{2}{3}\right)^{x}-12^{5-x}=0\) .Give BOTH an exact
	      answer and an approximate answer
	\item
	      Solve \(\frac{12+3e^{4-5x}}{5}=6\) Give BOTH an exact answer and an
	      approximate answer.
\end{enumerate}

29.(YOU TRY!) Solve \(\frac{10}{1+0.5e^{0.02t}}=6\) . Give BOTH an exact
answer and an approximate answer

\subsubsection{Solving Logarithmic
	Equations}\label{solving-logarithmic-equations}

\begin{enumerate}
	\def\labelenumi{\arabic{enumi}.}
	\setcounter{enumi}{30}
	\item
	      Solve \(\ln(x-4)=\ln(x+6)-\ln x\)
	\item
	      Solve \(\log_{2}(3x-4)=\log_{2}(x+2)\)
\end{enumerate}

Check answer(s):

Check answer(s):

Step1Given a logarithmicequation,isolate the logarithms on one side of
the equation

Step 2Use the properties oflogarithms to write the equationin the form
\(\log_b\) \(x=k\) ,where \(k\) is a constant

Step 3 Write the equation in exponential form. Step 4Solve the equation
from step3. Step 5 Check the potential solution(s) in the original
equation

32.Solve 4 \(\log_{3}(2t-7)=8\) . Check for extraneous solutions.

33.Solve \(\log(w+47)=2.6.\) Check for extraneous solutions. Give BOTH
an exact answer and an approximate answer rounded to two decimal places.

\begin{enumerate}
	\def\labelenumi{\arabic{enumi}.}
	\setcounter{enumi}{33}
	\item
	      Solve \(\ln2^{0.02x}=7\) . Check for extraneous solutions. Give BOTH
	      an exact answer and an approximate answer rounded totwo decimal places
	\item
	      Solve \(\log4^{x}=3\) . Check for extraneous solutions. Give BOTH an
	      exact answer and an approximate answer.
	\item
	      Solve \(\log_{2}x=3-\log_{2}(x-2)\) . Check for extraneous solutions.
	\item
	      Solve \(3\log(x+4)=6\) . Check for extraneous solutions.
	\item
	      Solve \(-5+\log_{2}(4-x)^{8}=9\) . Check for extraneous solutions.
	\item
	      Solve \(\log_{6}(x+8)-\log_{6}(x)=\log_{6}2\) . Check for extraneous
	      solutions.
\end{enumerate}

Solve Equations for a Specified Variable

\begin{enumerate}
	\def\labelenumi{\arabic{enumi}.}
	\setcounter{enumi}{39}
	\tightlist
	\item
	      Given \(P=100e^{kx}-100\) ,solve for \(x\) . (Used in geology)
\end{enumerate}

\includegraphics[width=\dimexpr\textwidth-0.5in\relax]{image.png}


	\begin{align}
		L = 8.8+5.1\log D \\  \\
	\end{align}


\subsection{Creating models for Exponential Growth and
	Decay}\label{creating-models-for-exponential-growth-and-decay}

\includegraphics[width=\dimexpr\textwidth-0.5in\relax]{./images/fCxAZ6g4Z54GeuG5VF2Wwd5Q9mCC9sWQB.png}

Notice,a also represent the \(y\) -intercept

That means, \(a\) is the initial value of \(y\) when \(k=0\)

\begin{enumerate}
	\def\labelenumi{\arabic{enumi}.}
	\setcounter{enumi}{41}
	\tightlist
	\item
	      The number of frogs in a wildlife preserve is increasing according to
	      the exponential model \(y=35e^{0.247x}\)
\end{enumerate}

\begin{itemize}
	\item
	      How did you know the equation represents an increasing population
	      model?

	      \begin{itemize}
		      \tightlist
		      \item
		            The exponent is is greater then 0 \(\implies\) increasing
	      \end{itemize}
	\item
	      What do \(x\) and y represent in terms of this problem?

	      \begin{itemize}
		      \tightlist
		      \item
		            \(y \rightarrow\) number of frogs in wildlife (\(y \in \mathrm{N}\))
		      \item
		            \(x \rightarrow\) time
	      \end{itemize}
	\item
	      Use the model to predict the number of frogs after 7 years, rounding
	      to the nearest frog. (Half a frog would be very sad)

	      \begin{itemize}
		      \tightlist
		      \item
		            \(y=35e^{0.247 \cdot 7} = 197.2255 = 197\) frogs
	      \end{itemize}
	\item
	      How long will it take for the population to reach 1000 frogs? (Round
	      to the nearest 2 decimal places.)
\end{itemize}


	\begin{align}
		1000= & 35e^{0.247x} \\ \frac{1000}{35} &= e^{0.247x} \\ \ln \frac{1000}{35} &= \ln e^{0.247x} \\ \ln \frac{1000}{35} &= 0.247x \\ \frac{\ln \frac{1000}{35}}{0.247} &=  x \\ x &= 13.57 \text{ years}\\
	\end{align}


\subsection{Investment Under Continuous
	Compounding}\label{investment-under-continuous-compounding}

\(A=Pe^{rt}\text{or}P(t)=P_0e^{rt}\)

\begin{itemize}
	\tightlist
	\item
	      P (Principal) stands for \textbf{initial amaount}
	\item
	      r(Rate) stands for \textbf{annual rate}
	\item
	      A (Amount) stands for \textbf{final amount in time}
\end{itemize}

\begin{enumerate}
	\def\labelenumi{\arabic{enumi}.}
	\setcounter{enumi}{42}
	\tightlist
	\item
	      Suppose that \(\$15,000\) is invested into a stock. After 3 years, the
	      value of the stock. account was \(\$19,356.92\) . Use the model
	      \(A=Pe^{rt}\) to determine the average return rate. \(r\) under
	      continuous compounding.
\end{enumerate}


	\begin{align}
		A & = Pe^{rt} \\ 19,356.92 &= 15,0000 e^{3r} \\ \ln \frac{19,356.92}{15,0000} &= \ln e^{3r} \\ \frac{\ln \dfrac{19,356.92}{15,0000}}{3} &= r \\
	\end{align}


\begin{enumerate}
	\def\labelenumi{\arabic{enumi}.}
	\setcounter{enumi}{43}
	\tightlist
	\item
	      On January 1, 2010, the population of California was approximately
	      37.3 million. On. January 1, 2019, the population was 40.0 million.
	      Let \(t=0\) represent the year 2010.
\end{enumerate}


	\begin{align}
		t = 0,\ 37.3\ G \\ t = 9,\ 40\ G \\ A=Pe^{rt} \\ 40 = 37.3 e^{9r} \\ \ln \frac{40}{37.3} = 9r \\ r = \frac{\ln \dfrac{40}{37.3}}{9} \\ r = 0.00777 \\ \implies A=Pe^{0.00777t} \\
	\end{align}


a.Write a function defined by \(P(t)=P_{0}e^{rt}\) to represent the
population of California \(P(t)\) in millions, t years after 2010.

\begin{enumerate}
	\def\labelenumi{\alph{enumi}.}
	\setcounter{enumi}{1}
	\tightlist
	\item
	      Use the function from part (a) to predict the population in 2025
\end{enumerate}


	\begin{align}
		A=Pe^{0.00777 \cdot 15} \\ A=37.3 \times 10^6 \cdot e^{0.00777t} \\
	\end{align}


\begin{enumerate}
	\def\labelenumi{\alph{enumi}.}
	\setcounter{enumi}{2}
	\tightlist
	\item
	      Use the function from part (a) to determine the year for which the
	      population of. California will be twice that of the year 2010.
\end{enumerate}


	\begin{align}
		A=Pe^{0.00777t} \\ 74.6 = 37.3 e^{0.00777t} \\ \ln \frac{74.6}{37.3} = 0.00777t \\ \frac{\ln \frac{74.6}{37.3}}{0.00777} = t \\ t = 89.2 \\ \implies \text{year} = 2010 + 89.2 \approx 2099 \\
	\end{align}


\subsubsection{\texorpdfstring{Investment with Interest Compounded
		\(\boldsymbol{n}\) times per
		year}{Investment with Interest Compounded \textbackslash boldsymbol\{n\} times per year}}\label{investment-with-interest-compounded-boldsymboln-times-per-year}

\(A=P\left(1+\frac{r}{n}\right)^{nt}\) or
\(P(t)=P\left(1+\frac{r}{n}\right)^{nt}\)

\begin{itemize}
	\tightlist
	\item
	      P \(\rightarrow\) \textbf{initial amount} (Principal)
	\item
	      r \(\rightarrow\) \textbf{interest rate} (Rate)
	\item
	      A \(\rightarrow\) \textbf{resultant amount} (Amount)
	\item
	      n \(\rightarrow\) \textbf{number of times compounded}
\end{itemize}

Compounding Option

\(n\) value

Annually

n = 1

Semi-annually

n = 2

Quarterly

n = 4

Monthly

n = 12

Daily

n = 365

Continuously

use the formula \(A=Pe^{rt}\)

\begin{enumerate}
	\def\labelenumi{\arabic{enumi}.}
	\setcounter{enumi}{44}
	\tightlist
	\item
	      Suppose that \(\$8,000\) is invested into an account, compounded
	      quarterly and it pays \(4.5\%\) per year.Approximately,how long does
	      it take the account to reach \(\$10,000?\)
\end{enumerate}

P = 8000

n = 4

r = 4.5\%

A = 10,000


	\begin{align}
		10,000 = 8000 (1+\frac{1.045}{4})^{4(t)} \\ \frac{10}{8} = \left ( \frac{809}{800}\right )^{4t} \\ \log \frac{10}{8} = 4t \log \frac{809}{800} \\ t = \frac{\log \dfrac{10}{8}}{4 \cdot \log \dfrac{809}{800}} \\
	\end{align}


46.A radioactive substance decays according to the exponential model
\(y=80e^{-0.0668x}\) where \(y\) is in milligrams and \(x\) is in years.

\begin{enumerate}
	\def\labelenumi{\alph{enumi}.}
	\tightlist
	\item
	      How much of the substance was there initially?
\end{enumerate}

80

b.What is the half-life of this substance?


	\begin{align}
		40 & = 80 e^{-0.0668x} \\ \ln \frac{1}{2} &= -0.0668x \\ x &= \frac{\ln \dfrac{1}{2}}{-0.0668} \\ x &= 10.38 \text{ years} \\
	\end{align}


\end{document}
