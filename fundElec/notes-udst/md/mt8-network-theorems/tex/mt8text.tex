\documentclass{book}

\usepackage{inctech}
\usepackage{circuitikz}

\begin{document}

\subsubsection{PROBLEMS}\label{problems}

\paragraph{SECTION 9.2 Superposition
Theorem}\label{section-9.2-superposition-theorem}

\begin{enumerate}
\def\labelenumi{\arabic{enumi}.}
\item
  \begin{enumerate}
  \def\labelenumii{\alph{enumii}.}
  \tightlist
  \item
    Using the superposition theorem, determine the current through the
    \(12 \Omega\) resistor of Fig. 9.119.
  \end{enumerate}

  \begin{enumerate}
  \def\labelenumii{\alph{enumii}.}
  \setcounter{enumii}{1}
  \item
    Convert both voltage sources to current sources and re-calculate the
    current to the \(12 \Omega\) resistor.
  \item
    How do the results of parts (a) and (b) compare?
  \end{enumerate}
\end{enumerate}

\begin{circuitikz}
  \draw
  (0,0) node[ground]{}
  to[battery1, l=$16V$, invert] (0,3)
  to[R, l=$4\Omega$] (3,3)
  to[R] (6,3)
  to[R, l=$12\Omega$] (6,0)
  -- (0,0);

  \draw
  (0,0)
  -- (3,0)
  to[battery1, l=$10V$] (3,3);

  \draw
  (3,3)
  to[R, l=$2\Omega$] (6,3);
\end{circuitikz}

FIG. 9.119 Problem 1.

\textbf{Ans}

\begin{circuitikz}
  \draw
  (0,0) node[ground]{}
  to[battery1, l=$16V$, invert] (0,3)
  to[R, l=$4\Omega$,name=R1] (3,3)
  to[R, i=$I_3$] (6,3)
  to[R, l=$12\Omega$,name=R3] (6,0)
  -- (0,0);

  \draw
  (0,0)
  -- (3,0)
  -- (3,3);

  \draw
  (3,3)
  to[R, l=$2\Omega$,name=R2] (6,3);

    % Add labels for the resistors
    \node[anchor=north] at (R1.south) {$R_1$};
    \node[anchor=north] at (R2.south) {$R_2$};
    \node[anchor=east] at (R3.south) {$R_3$};
\end{circuitikz}

%Beginning of stupidity \(\downarrow\)
%
%\begin{align*}
%  R_T &= R_1 + \frac{1}{\cfrac{1}{R_2}+\cfrac{1}{R_3}}  \\
%  R_T &= 4~\Omega + \frac{1}{\cfrac{1}{2~\Omega}+\cfrac{1}{12~\Omega}}  \\
%  R_T &= \frac{40}{7}~\Omega \\
%  I_S &= \frac{E}{R_T} \\
%  I_S &= \frac{16~V}{\cfrac{40}{7}~\Omega} = 2.8~A \\
%\end{align*}
%
%End of stupidity \(\uparrow\)

Current through \(12~\Omega\) resistor is 0 A. Because, the current
source is shorted.

Hence, \(I_3^\prime = 0~A\)

\begin{circuitikz}
  \draw
  (0,0) node[ground]{}
  -- (0,3)
  to[R, l=$4\Omega$,name=R1] (3,3)
  to[R] (6,3)
  to[R, l=$12\Omega$,name=R3] (6,0)
  -- (0,0);
  
  \draw
  (0,0)
  -- (3,0)
  to[short, i=$I_L$] (2,0);

  \draw
  (6,0)
  -- (3,0)
  to[short, i_>=$I_R$] (5,0);

  \draw
  (0,0)
  -- (3,0)
  to[battery1, l=$10V$,i=$I_S$] (3,3);

  \draw
  (3,3)
  to[R, l=$2\Omega$,name=R2] (6,3);
    % Add labels for the resistors
  \node[anchor=north] at (R1.south) {$R_1$};
  \node[anchor=north] at (R2.south) {$R_2$};
  \node[anchor=east] at (R3.south) {$R_3$};
\end{circuitikz}

\begin{align*}
  R_a &= R_2+R_3 = 2+12 = 14 \Omega\\
  R_T &= \frac{R_a R_1}{R_a+R_1} \\
  R_T &= \frac{14 \cdot 4}{14+4} = 3.11 \Omega \\ 
  I_S &= \frac{E}{R_T} \\
  I_S &= \frac{10}{3.11} = 3.22~A \\
  I_3 &= I_R = I_S \frac{R_1}{R_1+R_a} \\
  I_3 &= 3.22 \times \frac{4}{4+14} = 0.72~A \\
\end{align*}

\begin{enumerate}
\def\labelenumi{\arabic{enumi}.}
\item
  \begin{enumerate}
  \def\labelenumii{\alph{enumii}.}
  \item
    Using the superposition theorem, determine the voltage across the
    \(4.7 \Omega\) resistor of Fig. 9.120.
  \item
    Find the power delivered to the \(4.7 \Omega\) resistor due solely
    to the current source.
  \item
    Find the power delivered to the \(4.7 \Omega\) resistor due solely
    to the voltage source.
  \item
    Find the power delivered to the \(4.7 \Omega\) resistor using the
    voltage found in part (a).
  \item
    How do the results of part (d) compare with the sum of the results
    to parts (b) and (c)? Can the superposition theorem be applied to
    power levels?
  \end{enumerate}
\end{enumerate}

\includegraphics{image-11.png}

\begin{enumerate}
\def\labelenumi{\arabic{enumi}.}
\setcounter{enumi}{2}
\tightlist
\item
  Using the superposition theorem, determine the current through the
  \(56 \Omega\) resistor of Fig. 9.121.
\end{enumerate}

\includegraphics{image-12.png}

\end{document}