\documentclass{book}

\usepackage{blindtext}
\usepackage{inclink}
\usepackage{incpage}

\begin{document}

% Front matter
\frontmatter
\tableofcontents
\clearpage

% Main content
\mainmatter
\raggedright

\part{Basics}

\begin{itemize}
	\item \index{Conjunction}
	      
	      A conjunction is a word or words used to show the relationship between one notion and another notion. There are two main types of conjunction: the coordinative conjunction, which joins phrases of equal importance and rank, and the subordinative conjunction, which joins a phrase with another phrase that is dependent on it. 
	      
	\item \index{Adverb}
	      
	      An adverb is a word added to a verb, a participle, an adjective, or another adverb; and generally expresses time, place, degree, or manner
	      
	      Adverbs can modify a verb, a clause, adjective or a phrase
	      
\end{itemize}

\chapter{English Grammar}

English grammar refers to the set of rules and conventions that govern the structure of sentences, phrases, and words in the English language. It encompasses several key areas:

\section{Parts of Speech}
\begin{itemize}
	\item \textbf{Nouns:} Words that name people, places, things, or ideas (e.g., \textit{dog}, \textit{happiness}).
	\item \textbf{Pronouns:} Words that replace nouns (e.g., \textit{he}, \textit{they}).
	\item \textbf{Verbs:} Words that express actions or states of being (e.g., \textit{run}, \textit{is}).
	\item \textbf{Adjectives:} Words that describe nouns (e.g., \textit{beautiful}, \textit{red}).
	\item \textbf{Adverbs:} Words that describe verbs, adjectives, or other adverbs (e.g., \textit{quickly}, \textit{very}).
	\item \textbf{Prepositions:} Words that show relationships between nouns and other words (e.g., \textit{on}, \textit{under}).
	\item \textbf{Conjunctions:} Words that connect clauses, sentences, or words (e.g., \textit{and}, \textit{but}).
	\item \textbf{Interjections:} Words or phrases expressing emotion or exclamation (e.g., \textit{oh!}, \textit{wow!}).
\end{itemize}

\section{Sentence Structure}
Every sentence in English follows a general structure. This can be broken down into different components.

\subsection{Subject and Predicate}
\url{https://www.grammarly.com/blog/sentences/predicate/#:~:text=What%20is%20a%20compound%20predicate%3F}
Every sentence has a \textbf{subject} (what or who the sentence is about) and a \textbf{predicate} (what the subject does or is).

\subsection{Types of Sentences}
\begin{itemize}
	\item \textbf{Simple Sentences:} Contain one independent clause.
	\item \textbf{Compound Sentences:} Contain two or more independent clauses connected by conjunctions.
	\item \textbf{Complex Sentences:} Contain one independent clause and one or more dependent clauses.
	\item \textbf{Compound-Complex Sentences:} Contain at least two independent clauses and one or more dependent clauses.
\end{itemize}

\section{Tenses}
Tenses indicate the time of an action. There are three main tenses, with multiple variations:

\begin{itemize}
	\item \textbf{Past:} Actions completed in the past (e.g., \textit{I walked}).
	\item \textbf{Present:} Actions happening now or general truths (e.g., \textit{I walk}).
	\item \textbf{Future:} Actions that will happen (e.g., \textit{I will walk}).
\end{itemize}

Each tense can also take forms like simple, continuous, perfect, and perfect continuous.

\section{Subject-Verb Agreement}
In English, the subject and verb must agree in number and person:
\begin{itemize}
	\item \textit{He runs} (singular subject, singular verb).
	\item \textit{They run} (plural subject, plural verb).
\end{itemize}

\section{Modifiers}
Modifiers are words or phrases that provide additional information. The placement of modifiers matters:
\begin{itemize}
	\item \textit{The fast runner won.} (The adjective \textit{fast} modifies the noun \textit{runner}.)
	\item \textit{He only eats vegetables.} (The adverb \textit{only} modifies the verb \textit{eats}.)
	\item \textit{He eats only vegetables.} (Here, \textit{only} modifies \textit{vegetables}.)
\end{itemize}

\section{Punctuation}
Punctuation marks help clarify meaning and separate ideas:
\begin{itemize}
	\item \textbf{Period (.):} Ends declarative sentences.
	\item \textbf{Comma (,):} Separates elements in a list, clauses, or phrases.
	\item \textbf{Apostrophe ('):} Indicates possession or contractions.
	\item \textbf{Quotation Marks ("):} Enclose direct speech or quotes.
	\item \textbf{Colon (:)} and \textbf{Semicolon (;):} Link related ideas or separate complex items.
\end{itemize}

\section{Common Errors}
Here are some common grammar mistakes to watch for:
\begin{itemize}
	\item \textbf{Run-on Sentences:} Two independent clauses joined incorrectly.
	\item \textbf{Sentence Fragments:} Incomplete sentences.
	\item \textbf{Misplaced Modifiers:} Modifiers placed ambiguously.
	\item \textbf{Homophones:} Words that sound the same but differ in meaning or spelling (e.g., \textit{their} vs. \textit{there}).
\end{itemize}

\chapter{Tense, Aspect, and Mood (TAM)}

TAM stands for \textbf{Tense, Aspect, and Mood}, which are grammatical categories used to convey different layers of meaning about a verb in a sentence. Here’s a breakdown of each component:

\section*{1. Tense}
Indicates \emph{when} an action takes place—past, present, or future.  
\begin{itemize}
	\item Example: \emph{She runs} (present), \emph{She ran} (past), \emph{She will run} (future).
\end{itemize}

\section*{2. Aspect}
Describes the \emph{nature or structure} of the action, such as whether it is completed, ongoing, habitual, or momentary. Common aspects include:
\begin{itemize}
	\item \textbf{Perfect}: Completed action (\emph{She has run}).
	\item \textbf{Progressive}: Ongoing action (\emph{She is running}).
	\item \textbf{Habitual}: Repeated action (\emph{She runs every day}).
	\item \textbf{Perfective}: A single, completed event (\emph{She ran to the store}).
\end{itemize}

\section*{3. Mood}
Expresses the speaker’s attitude or the degree of certainty about the action. Common moods are:
\begin{itemize}
	\item \textbf{Indicative}: Factual or declarative statements (\emph{She runs}).
	\item \textbf{Subjunctive}: Hypothetical or desired situations (\emph{If she were to run}).
	\item \textbf{Imperative}: Commands (\emph{Run!}).
	\item \textbf{Conditional}: Actions dependent on conditions (\emph{She would run if it stopped raining}).
\end{itemize}

\section*{Interaction with Number, Person, and Gender}
In some languages, TAM is marked alongside grammatical categories like \textbf{number}, \textbf{person}, and \textbf{gender}. These grammatical features specify additional information about the subject of the verb:
\begin{itemize}
	\item \textbf{Number}: Whether the subject is singular or plural (\emph{I run} vs. \emph{We run}).
	\item \textbf{Person}: Whether the subject is first (\emph{I}), second (\emph{you}), or third person (\emph{he/she/it/they}).
	\item \textbf{Gender}: In languages like Arabic or French, verbs may agree with the subject’s gender (\emph{Il court} for "he runs" vs. \emph{Elle court} for "she runs").
\end{itemize}

Thus, while TAM deals with temporal and qualitative aspects of the verb, its forms may vary to align with these grammatical categories depending on the language.


\chapter{Manipulations}

INFLECTION is used as a general term that includes the CONJUGATION of verbs and the DECLENSION of nouns, pronouns and adjectives.

\chapter{Tenses}
\section{Simple \& Perfect Tenses}
\textbf{\href{https://promova.com/english-grammar/perfect-vs-simple-tenses-in-english}{[Reference]}}

The perfect tense expresses completed actions; the simple tense describes ongoing or imminent actions. Perfect tenses differ from past tenses by indicating ongoing relevance or recent completion.

The perfect and simple tenses are the two main types of tenses used in English. The other tenses in English, such as the future and the conditional, are formed using the perfect and simple tenses.

\part{Parts of Speech}

\chapter{Misc}

Pronouns

pronoun case structure

parallelism

sentence fragments/run-ons

adverbs, including modifiers and comparatives

verb tenses and regular and irregular verb forms

Subordination/coordination


\chapter{Nouns}

\textbf{\href{https://en.wikibooks.org/wiki/English_Grammar/Basic_Parts_of_Speech/Nouns}{[ref]}}

\section{Definitions\index{Noun!Definitions}:}
\begin{itemize}
	\item Noun\index{Noun}: any idea, quality, person, place or thing.
	\item Common noun\index{Noun!Definitions!Common noun}: any nonspecific person, place or thing.
	\item Proper noun\index{Noun!Definitions!Proper noun}: any specific person, place, living being, or thing.
	\item Countable noun\index{Noun!Definitions!Countable noun}
	\item Uncountable noun\index{Noun!Definitions!Uncountable noun} or Mass noun
	\item Concrete noun\index{Noun!Definitions!Concrete noun}
	\item Abstract noun\index{Noun!Definitions!Abstract noun}
	\item Collective noun\index{Noun!Definitions!Collective noun}: a collection of a specific type of elements
	\item Compound noun\index{Noun!Definitions!Compound noun}: made up of two or more words forming a unit idea.
\end{itemize}

\section{Properties\index{Noun!Properties}}

\begin{enumerate}
	\item Person
	\item Number
	\item Gender
	      \begin{itemize}
		      \item Masculine
		      \item Feminine
		      \item common
		      \item neuter
	      \end{itemize}
	\item Case
	      \begin{itemize}
		      \item Nominative
		            
		            nominative case can be either a subject or predicate but not both in a sentence.
		      \item Objective
		            
		            can be used as object of the verb or object of the preposition
		      \item Possessive
		            
		            usually formed by adding an apostrophe (') or an apostrophe s ('s)
	      \end{itemize}
\end{enumerate}

\chapter{Pronouns\index{Pronouns}}

A pronoun replaces a noun in a sentence. The noun that is replaced is called an antecedent. 

\chapter{Verbs\index{Verbs}}

References:

\url{https://www.grammarly.com/blog/parts-of-speech/verb-tenses/}

A verb is a word or group of words expressing an action or a state.

\section{Conjugation}

A verb form\label{Verbs!verb form} is a way in which a verb is shaped or modified in order to suit the context that speaks about an action that is performed at a specific time.

Types:
\begin{itemize}
	\item root verb
	\item third person singular present form of verb
	\item present participle
	\item simple past participle
	\item past participle
\end{itemize}

\section{Types}

\begin{itemize}
	\item transitive and intransitive verbs
	      
	      \url{https://www.grammarly.com/blog/parts-of-speech/transitive-and-intransitive-verbs/}
\end{itemize}

\section{Verbals\index{Verbs!Verbals}}

A verbal (is a form of a verb or is a derivation from a verb ) that functions as some other part of speech in a sentence

\subsection{Infinitive\index{Verbs!Verbals!Infinitive}}

\url{https://www.grammarly.com/blog/grammar/infinitives/}

the basic form of a verb, without an inflection binding it to a particular subject or tense

\subsection{Gerunds\index{Verbs!Verbals!Gerunds}}

Gerunds are nouns built from a verb with an ing suffix.

Uses:
\begin{itemize}
	\item subject of a sentence
	\item an object
	\item an object of preposition
	\item to complement a subject.
\end{itemize}

\subsection{Participles\index{Verbs!Verbals!Participles}}

Participles are forms of verbs which are used as adjectives.

Structure:

\begin{table}[h!] % Optional position specifier
	\setlength\tabcolsep{3pt} % Adjust horizontal padding between columns
	\renewcommand{\arraystretch}{1.2} % Adjust row height for better readability
	\centering % Center the table on the page
	\begin{tabular}{@{}>{\raggedright\arraybackslash}p{2.5cm}%
		>{\raggedright\arraybackslash}p{3cm}%
		>{\raggedright\arraybackslash}p{6cm}@{}}
		\toprule
		\textbf{Tense} & \textbf{Inflection}           & \textbf{Typical Meaning}                             \\
		\midrule
		Present        & ~root + ing                   & Continuance of the being, action, or passion         \\
		Past           & ~root + ed/en                 & Completion of the being, action, or passion          \\
		Past Perfect   & ~had/having + past participle & Previous completion of the being, action, or passion \\
		\bottomrule
	\end{tabular}
	\label{tab:tenses}
\end{table}

\section{Phrasal verbs}

References:
\url{https://www.grammarly.com/blog/parts-of-speech/common-phrasal-verbs/}

A preposition is a word used to express some relation of different things or thoughts to each other, and is generally placed before a noun or a pronoun

\chapter{idioms}

An idiom is a phrase that, when taken as a whole, has a meaning you wouldn’t be able to deduce from the meanings of the individual words.

Types:
\begin{itemize}
	\item Pure idiom
	\item Binomial idiom
	\item Partial idiom
	\item Prepositional idiom
\end{itemize}

\chapter{Particles\index{Particles}}

Word that does not change its form through inflection and does not fit easily into the established system of parts of speech.

\chapter{Prepositions\index{Prepositions}}

A preposition is a word or group of words used before a noun, pronoun, or noun phrase to show direction, time, place, location, spatial relationships, or to introduce an object.

"The paper lies before me on the desk."

In that sentence, before is the preposition, me is the governed term of a preposition, "before me" is a prepositional phrase, and the verb lies is the prior term of a preposition. "On the desk" is the other prepositional phrase, and lies is its prior term.

To a preposition, the prior term may be a noun, an adjective, a pronoun, a verb, a participle, or an adverb; and the governed term may be a noun, a pronoun, a pronominal adjective, an infinitive verb, or a participle. 

\chapter{Adverbs}

An adverb is a word added to a verb, a participle, an adjective, or another adverb; and generally expresses time, place, degree, or manner: as,

"They are now here, studying very diligently."

Adverbs can modify a verb, a clause, adjective or a phrase. 

\chapter{Adjectives\index{Adjectives}}

An adjective is a part of speech used as a modifier that describes a noun or pronoun.

\chapter{Conjuctions}

A conjunction is a word used to connect words or sentences in construction, and to show the dependence of the terms so connected

\begin{itemize}
	\item copulative conjunction is a conjunction that denotes an addition, a cause, a consequence, or a supposition
	\item disjunctive conjunction is a conjunction that denotes opposition of meaning
	\item corresponsive conjunctions are those which are used in pairs, so that one refers or answers to the other.
\end{itemize}

\begin{tabular}{|l|l|l|}
	\hline
	copulative & disjunctive     & corresponsive \\ 
	\hline
	and        & although        & and; as       \\ 
	as         & but             & as; as        \\ 
	because    & either          & both          \\ 
	both       & except          & nor; whether  \\ 
	even       & lest            & or; neither   \\ 
	for        & neither         & or; though    \\ 
	if         & nor             & so; if        \\ 
	seeing     & notwithstanding & then; either  \\ 
	since      & or              & yet           \\ 
	so         & provided        & yet; although \\ 
	that       & save            & ~             \\ 
	then       & than            & ~             \\ 
	~          & though          & ~             \\ 
	~          & unless          & ~             \\ 
	~          & whereas         & ~             \\ 
	~          & whether         & ~             \\ 
	~          & yet             & ~             \\
	\hline
\end{tabular}




\chapter{Connectives}



\chapter{Sequences}

\chapter{Interjection\index{Interjection}}

An interjection is a word or group of words that is used to express surprise, fear, pain or other emotions. It is not grammatically related to other words in a sentence, so it functions independently. It may be followed by an exclamation point, or a comma when part of a complete sentence.

\chapter{Modifiers: Are All Modifiers Adverbs?}

Modifiers are words or phrases that provide additional information about other words in a sentence. They can describe, limit, or qualify the meaning of the words they modify. While many modifiers are adverbs, not all modifiers are adverbs. Modifiers can be broadly classified into two types: adjectives and adverbs, depending on what they modify.

\section{Adjectives as Modifiers}
Adjectives are words that modify or describe nouns (people, places, things, or ideas). They provide additional information about a noun, such as its quality, quantity, or state.

\subsection{Examples of Adjectives Modifying Nouns:}
\begin{itemize}
	\item \textbf{The \textit{blue} sky} (The adjective \textit{blue} modifies the noun \textit{sky}.)
	\item \textbf{She wore a \textit{beautiful} dress} (The adjective \textit{beautiful} modifies the noun \textit{dress}.)
	\item \textbf{It was a \textit{long} journey} (The adjective \textit{long} modifies the noun \textit{journey}.)
\end{itemize}

In these examples, the adjectives give us more information about the nouns they modify.

\section{Adverbs as Modifiers}
Adverbs, on the other hand, modify verbs, adjectives, or other adverbs. They describe how, when, where, or to what extent an action occurs, or they provide more information about the adjective or adverb they modify.

\subsection{Examples of Adverbs Modifying Verbs:}
\begin{itemize}
	\item \textbf{She sings \textit{beautifully}.} (The adverb \textit{beautifully} modifies the verb \textit{sings}.)
	\item \textbf{He runs \textit{quickly}.} (The adverb \textit{quickly} modifies the verb \textit{runs}.)
	\item \textbf{They worked \textit{hard} yesterday.} (The adverb \textit{hard} modifies the verb \textit{worked}.)
\end{itemize}

\subsection{Examples of Adverbs Modifying Adjectives:}
\begin{itemize}
	\item \textbf{She is \textit{very} talented.} (The adverb \textit{very} modifies the adjective \textit{talented}.)
	\item \textbf{The movie was \textit{incredibly} boring.} (The adverb \textit{incredibly} modifies the adjective \textit{boring}.)
	\item \textbf{He is \textit{extremely} tired.} (The adverb \textit{extremely} modifies the adjective \textit{tired}.)
\end{itemize}

\subsection{Examples of Adverbs Modifying Other Adverbs:}
\begin{itemize}
	\item \textbf{She sings \textit{quite} beautifully.} (The adverb \textit{quite} modifies the adverb \textit{beautifully}.)
	\item \textbf{He runs \textit{too} quickly.} (The adverb \textit{too} modifies the adverb \textit{quickly}.)
	\item \textbf{They worked \textit{very} hard.} (The adverb \textit{very} modifies the adverb \textit{hard}.)
\end{itemize}

\section{Are All Modifiers Adverbs?}
No, not all modifiers are adverbs. While adverbs modify verbs, adjectives, and other adverbs, adjectives are also modifiers, but they modify nouns. In addition to adjectives and adverbs, other modifiers can include determiners and phrases that provide additional details about nouns or actions.

\subsection{Key Differences Between Adjectives and Adverbs as Modifiers:}
\begin{itemize}
	\item \textbf{Adjectives modify nouns.} They provide more information about the characteristics or qualities of a noun.
	\item \textbf{Adverbs modify verbs, adjectives, or other adverbs.} They provide more information about the manner, time, degree, or frequency of an action, or they describe the intensity of a quality or another adverb.
\end{itemize}

For example:
\begin{itemize}
	\item \textbf{Adjective:} \textit{The \textit{blue} sky is beautiful.} (The adjective \textit{blue} modifies the noun \textit{sky}.)
	\item \textbf{Adverb:} \textit{She sings \textit{beautifully}.} (The adverb \textit{beautifully} modifies the verb \textit{sings}.)
\end{itemize}

\section{Conclusion}
Modifiers play an essential role in the English language by adding detail and enhancing meaning. While many modifiers are adverbs, adjectives are also crucial modifiers that provide essential information about nouns. Understanding the distinction between these types of modifiers is important for constructing clear and accurate sentences.

\part{Words, Phrases \& Clauses }

\chapter{Compound words}

\href{https://www.grammarly.com/blog/grammar/open-and-closed-compound-words/}{General}

\href{https://www.grammarly.com/blog/punctuation-capitalization/hyphen-with-compound-modifiers/}{Hyphen With Compound Modifiers}

\part{Sentences}

\chapter{Proving a sentence is grammatically correct}

Proving that a sentence is grammatically correct involves checking it against the established rules of grammar for the language in question. Here’s how you can systematically verify grammatical correctness:

\begin{itemize}
	\item Subject-Verb Agreement
	      
	      Ensure the subject and verb agree in number and person.
	      
	      ex: "She runs every morning." (Correct: Singular subject with singular verb)
	\item Sentence Structure
	      
	      Confirm the sentence has at least one independent clause: a subject, a verb, and expresses a complete thought.
	      
	      ex: "The cat sat on the mat." (Subject: The cat; Verb: sat; Object: the mat)
	\item Punctuation
	      
	      Check for proper punctuation, including periods, commas, colons, and semicolons.
	      
	      ex: "Let's eat, Grandma!" vs. "Let's eat Grandma!" (The comma changes the meaning.)
	\item Word Order
	      
	      Verify that the sentence follows the typical word order for the language (e.g., Subject-Verb-Object in English).
	      
	      ex: "I ate the cake." (Correct word order)
	\item Consistency in Tense
	      
	      Ensure the verb tenses are consistent unless a change in tense is required by the context.
	      
	      ex: "She was reading when he called." (Past continuous for an ongoing action and simple past for an interruption)
	\item Modifiers
	      
	      Check that modifiers are placed close to the word they modify to avoid confusion or ambiguity.
	      
	      ex: "The boy who was crying loudly left the room." (Correct modifier placement)
	\item Prepositions
	      
	      Ensure prepositions are used correctly based on the context.
	      
	      ex: "She is good at math." (Correct use of "at")
	\item Parallelism
	      
	      Look for parallel structure in lists or comparisons.
	      
	      ex: "She likes swimming, running, and hiking." (Consistent gerund forms)
	\item Articles and Determiners
	      
	      Check that articles (a, an, the) and determiners (this, that, these) are used appropriately.
	      
	      ex: "A cat is on the roof." (Correct use of "a" with a singular noun)
	\item Pronouns and Antecedents
	      
	      Verify that pronouns clearly refer to their antecedents and agree in number and gender.
	      
	      ex: "Each student must bring their own pencil." (Correct: "Each" is singular, but "their" is accepted in modern English)
\end{itemize}

By systematically evaluating the sentence against these rules, you can determine if it is grammatically correct. If you're unsure about a specific sentence, feel free to share it, and I can help analyze it!

\chapter{Specific elements of sentences}

A sentences\index{Sentence} is a group of words with a subject and predicat and expresses a complete thought. That sounds similar to clauses. But that's incorrect. "and expresses a complete thought" is a difference and a sentence at minimum is in the form of a clause but can contain multiple clauses as long as it has a subject and a predicate.

\section{Phrases\index{Phrases}}

A phrase is a group of words which contains neither a subject nor a verb. (It may, however, contain a verbal form such as an infinitive, a participle, or a gerund.)

\section{Clauses\index{Clauses}}

It is below the sentence in rank and in traditional grammar said to consist of a subject and predicate. There are two kinds of clauses: independent and dependent.

\section{Predicate\index{Predicate}}

The part of a sentence or clause containing a verb and stating something about the subject

\section{Subject Complements\index{Subject Complements}}

A subject complement is a word or phrase that appears after a linking verb in a sentence and is closely related to the sentence’s subject—identifying, defining, or describing it.

A subject complement’s job, along with a linking verb, is to clarify the subject of a sentence.

Subject complements never appear without linking verbs, and linking verbs never appear without them.\textbf{\href{https://www.grammarly.com/blog/grammar/subject-complement/}{[ref]}}

\subsection{Object\index{Object}}

Direct object: The noun that receives the action.

Indirect object: The noun that receives the direct object.

\chapter{Forms\index{Sentence!Forms}}

\url{https://www.grammarly.com/blog/sentences/sentence-structure/}

\begin{itemize}
	\item Simple sentences\index{Sentence!Forms!Simple sentences}: 1 independent clause
	\item Compound sentences\index{Sentence!Forms!Compound sentences}: 2 or more independent clauses
	\item Complex sentences\index{Sentence!Forms!Complex sentences}:1 independent clause + 1 or more subordinate clauses
	\item Compound-Complex sentences\index{Sentence!Forms!Compound-Complex sentences}: 2 or more independent clauses + 1 or more subordinate clauses
\end{itemize}

\chapter{Types\index{Sentence!Types}}

\url{https://www.grammarly.com/blog/sentences/kinds-of-sentences/}

\begin{itemize}
	\item declarative
	\item interrogative
	\item imperative
	      Single word sentences are possible
	\item exclamatory
\end{itemize}

\chapter{Phrases}
\url{https://www.grammarly.com/blog/sentences/phrases/}

Phrases are of 2 kind: grammatical phrases and common phrases

\section{Grammatical Phrases}
A grammatical phrase is a collection of words working together as a unit.

A grammatical phrase can clarify any part of speech—the key here is that all a phrase does is provide some detail;

\subsection{Appositive phrases}
An appositive is a noun phrase that comes after another noun phrase (its antecedent) to provide extra information about it. ex: "The student, Fairoos, is writing a latex file"

\section{Common Phrases}
Common phrases, are pieces of figurative language that rely on the listener’s familiarity with them to be understood

Here:
\begin{itemize}
	\item Appositive: Fairoos
	\item Antecedant: The student
\end{itemize}

\chapter{Clauses}
\url{https://www.grammarly.com/blog/grammar/clauses/}

Every sentence you write includes at least one independent clause

\chapter{Structure}
\begin{tabular}{c}
	\toprule
	Subject + Phrase      \\
	\midrule
	Subject + Verb-phrase \\
	\\
	\bottomrule
\end{tabular}

\subsection{Basic Rules}
\begin{itemize}
	\item Capitalize the first letter of the first word in a sentence.
	\item End a sentence with a period, question mark, exclamation point, or quotation marks.
	\item Most of the time, the subject of the sentence comes first, the verb comes second, and the objects come last. (Subject -> Verb -> Object)
	\item If the subject is singular, the verb must also be singular. If the subject is plural, the verb must be plural. This is known as subject-verb agreement.
\end{itemize}

\chapter{Punctuation}
\url{https://www.grammarly.com/blog/punctuation-capitalization/punctuation/}

\section{Period}
Use: to end a declarative sentence.

\section{Comma}

\url{https://www.grammarly.com/blog/punctuation-capitalization/comma/}

\url{https://www.grammarly.com/blog/punctuation-capitalization/comma-before-and/}

Uses:
\begin{itemize}
	\item indicates a pause in a sentence, either between phrases, clauses, or items in a list
\end{itemize}

the points where you’d pause in a spoken sentence aren’t always where you’d use a comma in a written sentence.

\section{Apostrophe (')}
Uses:
\begin{itemize}
	\item Creating possessive nouns
	\item Combining words into contractions
	\item more casually, apostrophes are used to shorten words (government becomes gov’t and the 1970s becomes the ’70s)
	\item in quotes to show the speaker has shortened a word, for example: nothin’.
	\item only time an apostrophe is used to pluralize a noun is when the noun being pluralized is a lowercase letter. For example: Mind your p’s and q’s.
\end{itemize}
Don'ts:
\begin{itemize}
	\item Most of the time, they are not used to pluralize nouns
	\item use them when you’re referring to a decade numerically (correct: the 1990s, incorrect: the 1990’s)
	\item use them when the last letter follows an apostrophe (correct: don’ts, incorrect: don’t’s)
	\item use them when describing a group of people (correct: the Chens are coming to dinner, incorrect: the Chen’s are coming to dinner)
\end{itemize}

\section{Ellipses (. . .)}

Use:
\begin{itemize}
	\item to show that information has been omitted from a quote, usually to shorten it
	\item they’re also used to build suspense
	\item show a speaker’s voice is trailing off or faltering
	\item represent incomplete thoughts
	\item to indicate pauses
	\item voices or thoughts fading away
	\item Emotion of speechlessness
\end{itemize}

\chapter{Proofreading}

\url{https://grammarist.com/editing/what-is-proofreading/}

\section{Tips}

\begin{itemize}
	\item Make a list of your personal bugaboos.
	\item Read it. Wait a minute. Then read it again.
	\item Read backward.
	\item Change the view.
	\item Read it out loud
\end{itemize}

\chapter{Subject \& Object}

\url{https://www.grammarly.com/blog/parts-of-speech/the-basics-on-subject-and-object-pronouns-b/}

\chapter{Subject-Verb Agreement}

\url{https://www.grammarly.com/blog/grammar/grammar-basics-what-is-subject-verb-agreement/}

\chapter{Grammatical Structures in English}

In English, various grammatical structures are used to form different types of sentences. Here are some common structures:

\section{Simple Sentences}

\subsection{Subject + Main Verb}
This is the most basic sentence structure, where the subject is followed by the main verb.
\begin{itemize}
	\item Example: \textit{She runs.}
	\item Structure: \texttt{[Subject] + [Main Verb].}
\end{itemize}

\section{Sentences with Direct Objects}

\subsection{Subject + Main Verb + Direct Object}
This structure includes a subject, a main verb, and a direct object.
\begin{itemize}
	\item Example: \textit{She reads a book.}
	\item Structure: \texttt{[Subject] + [Main Verb] + [Direct Object].}
\end{itemize}

\section{Sentences with Indirect Objects}

\subsection{Subject + Main Verb + Indirect Object + Direct Object}
Here, the sentence has a subject, a main verb, an indirect object, and a direct object.
\begin{itemize}
	\item Example: \textit{She gives him a gift.}
	\item Structure: \texttt{[Subject] + [Main Verb] + [Indirect Object] + [Direct Object].}
\end{itemize}

\section{Sentences with Modifiers}

\subsection{Subject + Main Verb + Object + Modifier}
This structure includes a subject, main verb, object, and a modifier (which can be an adjective or adverb).
\begin{itemize}
	\item Example: \textit{She reads the book quickly.}
	\item Structure: \texttt{[Subject] + [Main Verb] + [Object] + [Modifier].}
\end{itemize}

\section{Sentences with Infinitives}

\subsection{Subject + Main Verb + Infinitive}
In this structure, the verb is followed by an infinitive verb phrase (to + verb).
\begin{itemize}
	\item Example: \textit{She wants to study.}
	\item Structure: \texttt{[Subject] + [Main Verb] + [Infinitive].}
\end{itemize}

\subsection{Subject + Main Verb + Infinitive + Object}
This structure includes an infinitive verb followed by an object.
\begin{itemize}
	\item Example: \textit{She wants to eat dinner.}
	\item Structure: \texttt{[Subject] + [Main Verb] + [Infinitive] + [Object].}
\end{itemize}

\section{Modal Verbs}

\subsection{Subject + Modal Verb + Main Verb}
This structure uses a modal verb (can, should, might, etc.) followed by the main verb in its base form.
\begin{itemize}
	\item Example: \textit{She can run.}
	\item Structure: \texttt{[Subject] + [Modal Verb] + [Main Verb].}
\end{itemize}

\subsection{Subject + Modal Verb + Main Verb + Object}
Here, a modal verb is used with the main verb and an object.
\begin{itemize}
	\item Example: \textit{She can read a book.}
	\item Structure: \texttt{[Subject] + [Modal Verb] + [Main Verb] + [Object].}
\end{itemize}

\section{Sentences with Adverbs}

\subsection{Subject + Main Verb + Adverb}
In this structure, an adverb modifies the main verb or the entire sentence.
\begin{itemize}
	\item Example: \textit{She runs quickly.}
	\item Structure: \texttt{[Subject] + [Main Verb] + [Adverb].}
\end{itemize}

\subsection{Subject + Main Verb + Object + Adverb}
Here, the adverb modifies either the object or the verb in the sentence.
\begin{itemize}
	\item Example: \textit{She reads the book slowly.}
	\item Structure: \texttt{[Subject] + [Main Verb] + [Object] + [Adverb].}
\end{itemize}

\section{8. Complex Sentences}

\subsection{Subject + Main Verb + Direct Object + Dependent Clause}
In complex sentences, a dependent clause follows the main clause.
\begin{itemize}
	\item Example: \textit{She reads because she enjoys it.}
	\item Structure: \texttt{[Subject] + [Main Verb] + [Direct Object] + [Dependent Clause].}
\end{itemize}

\subsection{Subject + Main Verb + Infinitive + Dependent Clause}
Here, the infinitive is followed by a dependent clause.
\begin{itemize}
	\item Example: \textit{She wants to study because she loves learning.}
	\item Structure: \texttt{[Subject] + [Main Verb] + [Infinitive] + [Dependent Clause].}
\end{itemize}

\section{Compound Sentences}

\subsection{Independent Clause + Coordinating Conjunction + Independent Clause}
A compound sentence consists of two independent clauses joined by a coordinating conjunction.
\begin{itemize}
	\item Example: \textit{She reads, and he writes.}
	\item Structure: \texttt{[Independent Clause] + [Coordinating Conjunction] + [Independent Clause].}
\end{itemize}

\chapter{Subject}

\url{https://www.grammarly.com/blog/grammar/compound-subject/}

\backmatter

\chapter{References:}

\section{Books}

\href{https://en.wikibooks.org/wiki/English_in_Use}{English in Use}

\section{Websites}

Grammarly

\printindex\label{sec:Index}
\end{document}
