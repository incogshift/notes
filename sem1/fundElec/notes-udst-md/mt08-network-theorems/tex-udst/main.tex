\documentclass{article}
\usepackage{amsmath}
\usepackage{circuitikz}

\begin{document}

\section*{SUPERPOSITION THEOREM}

\subsection*{EXAMPLE 9.1}

Using the superposition theorem, determine current $I_1$ for the network in Fig. 9.2.

\begin{figure}[h!]
    \centering
    \begin{circuitikz}
        \draw
        (0,0) node[ground]{}
        (0,0) -- (0,3) to[battery, l=$E$, l_=$30V$] (0,3)
        -- (3,3) node{}
        (3,3) to[ammeter, name=I] (3,0)
        (3,3) -- (5,3) to[R=$R_1$, l=$6\Omega$] (5,0) -- (0,0)
        (3,0) -- (0,0)
        (5,3) node[above]{$I_1$}
        (3,3) to[current source, l=$3A$] (3,1.5);
    \end{circuitikz}
    \caption{Two-source network to be analyzed using the superposition theorem in Example 9.1.}
    \label{fig:9.2}
\end{figure}

\noindent
\textbf{Solution:} Since two sources are present, there are two networks to be analyzed.

\noindent
Due to the open circuit, resistor R1 is in parallel with the voltage source E. The voltage across the resistor is the applied voltage, and current I1 is determined by:

$I_{1}^{\prime\prime}=\frac{R_{sc}I}{R_{sc}+R_{1}}=\frac{(0~\Omega)I}{0~\Omega+6~\Omega}=0~A$

\begin{figure}[h!]
    \centering
    \begin{circuitikz}
        \draw
        (0,0) node[ground]{}
        (0,0) -- (0,3) to[battery, l=$E$, l_=$30V$] (0,3)
        -- (3,3)
        (3,3) to[short, name=I] (3,0)
        (3,3) -- (5,3) to[R=$R_1$, l=$6\Omega$] (5,0) -- (0,0);
        (5,3) node[above]{$I_1'$}
    \end{circuitikz}
    \caption{Determining the effect of the 30 V supply on the current I1 in Fig. 9.2.}
    \label{fig:9.3}
\end{figure}
\noindent
Now for the contribution due to the current source. Setting the voltage source to zero volts results in the network in Fig. 9.4,

$I_{1}^{\prime}=\frac{R_{sc}I}{R_{sc}+R_{1}}=\frac{(0~\Omega)I}{0~\Omega+6~\Omega}=0~A$

\begin{figure}[h!]
    \centering
    \begin{circuitikz}
        \draw
        (0,0) node[ground]{}
        (0,3) to[current source, l=$3A$, name=I] (0,0)
        (0,3) -- (3,3) -- (5,3) to[R=$R_1$, l=$6\Omega$] (5,0) -- (0,0);
        (5,3) node[above]{$I_1''$}
    \end{circuitikz}
    \caption{Determining the effect of the 3 A current source on the current $I_1$ in Fig. 9.2.}
    \label{fig:9.4}
\end{figure}

\noindent
Since $I_{1}^{\prime}$ and $I_{1}^{\prime\prime}$ have the same defined direction in Figs. 9.3 and 9.4, the total current is defined by

$I_{1}=I_{1}^{\prime}+I_{1}^{\prime\prime}=5~A+0~A=5~A$

\subsection*{EXAMPLE 9.2}

Using the superposition theorem, determine the current through the \(12~\Omega\) resistor in Fig. 9.5.

\begin{figure}[h!]
    \centering
    \begin{circuitikz}
        \draw
        (0,0) node[ground]{}
        (0,4) node[battery, l=$E_1$, l_=$54V$] (E1)
        (0,0) -- (0,4) node[battery, l=$E_1$, l_=$54V$] (E1)
        (E1) -- (4,4) to[R=$R_1$, l=$24\Omega$] (4,0) -- (0,0)
        (4,4) -- (4,2) to[R=$R_2$, l=$12\Omega$, i=$I_2$] (4,0)
        (4,4) -- (8,4) to[R=$R_3$, l=$4\Omega$] (8,0)
        (8,4) node[battery, l=$E_2$, l_=$48V$] (E2)
        (8,0) -- (E2);
    \end{circuitikz}
    \caption{Using the superposition theorem to determine the current through the 12 $\Omega$ resistor (Example 9.2).}
    \label{fig:9.5}
\end{figure}

\noindent
Considering the effects of the 54 V source requires replacing the 48 V source by a short-circuit.

\begin{figure}[h!]
    \centering
    \begin{circuitikz}
        \draw
        (0,0) node[ground]{}
        (0,4) node[battery, l=$E_1$, l_=$54V$] (E1)
        (0,0) -- (E1)
        (E1) -- (4,4) to[R=$R_1$, l=$24\Omega$] (4,0) -- (0,0)
        (4,4) -- (4,2) to[R=$R_2$, l=$12\Omega$, i=$I_2'$] (4,0)
        (4,4) -- (8,4) to[R=$R_3$, l=$4\Omega$] (8,0) -- (4,0);
    \end{circuitikz}
    \caption{Using the superposition theorem to determine the effect of the 54 V voltage source on current $I_2$ in Fig. 9.5.}
    \label{fig:9.6}
\end{figure}

\noindent
The total resistance seen by the source is therefore

$R_{T}=R_{1}+R_{2}||R_{3}=24~\Omega+12~\Omega||4~\Omega=24~\Omega+3~\Omega=27~\Omega$

and the source current is

$I_{s}=\frac{E_{1}}{R_{T}}=\frac{54~V}{27~\Omega}=2~A$

\noindent
Using the current divider rule results in the contribution to $I_{2}$ due to the 54 V source:

$I_{2}=\frac{R_{3}I_{s}}{R_{3}+R_{2}}=\frac{(4\Omega)(2A)}{4~\Omega+12~\Omega}=0.5~A$

\noindent
Considering the effects of the 48 V source requires replacing the 54 V source by a short-circuit.

\begin{figure}[h!]
    \centering
    \begin{circuitikz}
        \draw
        (0,0) node[ground]{}
        (0,4) to[R=$R_1$, l=$24\Omega$] (0,0)
        (0,4) -- (4,4) -- (4,2) to[R=$R_2$, l=$12\Omega$, i=$I_2''$] (4,0)
        (4,4) -- (8,4) node[battery, l=$E_2$, l_=$48V$] (E2)
        (8,0) -- (E2) -- (4,0)
        (8,4) to[R=$R_3$, l=$4\Omega$] (8,0);
    \end{circuitikz}
    \caption{Using the superposition theorem to determine the effect of the 48 V voltage source on current $I_2$ in Fig. 9.5.}
    \label{fig:9.7}
\end{figure}

\noindent
Therefore, the total resistance seen by the 48 V source is

$R_{T}=R_{3}+R_{2}||R_{1}=4~\Omega+12~\Omega||24~\Omega=4~\Omega+8~\Omega=12~\Omega$

and the source current is

$I_{s}=\frac{E_{2}}{R_{T}}=\frac{48~V}{12~\Omega}=4~A$

\noindent
Applying the current divider rule results in

$I_{2}^{\prime\prime}=\frac{R_{1}(I_{s})}{R_{1}+R_{2}}=\frac{(24~\Omega)(4A)}{24~\Omega+12~\Omega}=2.67~A$

\noindent
It is now important to realize that current $I_2$ due to each source has a different direction, as shown in Fig. 9.8. The net current therefore is the difference of the two and the direction of the larger as follows:

$I_{2}=I_{2}^{\prime\prime}-I_{2}^{\prime}=2.67A-0.5~A=2.17~A$

\begin{figure}[h!]
    \centering
    \begin{circuitikz}
        \draw
        (0,0) node[ground]{}
        (0,2) to[R=$R_2$, l=$12\Omega$, i=$I_2'=0.5A$] (0,0)
        (2,2) to[R=$R_2$, l=$12\Omega$, i=$I_2''=2.67A$, i_=$I_2=2.17A$] (2,0);
    \end{circuitikz}
    \caption{Using the results of Figs. 9.6 and 9.7 to determine current $I_2$ for the network in Fig. 9.5.}
    \label{fig:9.8}
\end{figure}

\subsection*{EXAMPLE 9.3}

a. Using the superposition theorem, determine the current through resistor $R_2$ for the network in Fig. 9.9.

b. Demonstrate that the superposition theorem is not applicable to power levels.

\noindent
\textbf{Solutions:}

a. In order to determine the effect of the 36 V voltage source, the current source must be replaced by an open-circuit equivalent as shown in Fig. 9.10. The result is a simple series circuit with a current equal to

$I_{2}=\frac{E}{R_{T}}=\frac{E}{R_{1}+R_{2}}=\frac{36V}{12~\Omega+6~\Omega}=\frac{36~V}{18~\Omega}=2~A$

\begin{figure}[h!]
    \centering
    \begin{circuitikz}
        \draw
        (0,0) node[ground]{}
        (0,4) node[battery, l=$E$, l_=$36V$] (E)
        (0,0) -- (E)
        (E) -- (4,4) to[R=$R_1$, l=$12\Omega$] (4,0) -- (0,0)
        (4,4) -- (4,2) to[R=$R_2$, l=$6\Omega$, i=$I_2$] (4,0)
        (4,4) -- (8,4) to[current source, l=$9A$] (8,0);
        (8,4) node[right]{};
    \end{circuitikz}
    \caption{Network to be analyzed in Example 9.3 using the superposition theorem.}
    \label{fig:9.9}
\end{figure}

\begin{figure}[h!]
    \centering
    \begin{circuitikz}
        \draw
        (0,0) node[ground]{}
        (0,4) node[battery, l=$E$, l_=$36V$] (E)
        (0,0) -- (E)
        (E) -- (4,4) to[R=$R_1$, l=$12\Omega$] (4,0) -- (0,0)
        (4,4) -- (4,2) to[R=$R_2$, l=$6\Omega$, i=$I_2'$] (4,0);
    \end{circuitikz}
    \caption{Current source replaced by open circuit.}
    \label{fig:9.10}
\end{figure}

\noindent
Examining the effect of the 9 A current source requires replacing the 36 V voltage source by a short-circuit equivalent as shown in Fig. 9.11. The result is a parallel combination of resistors $R_1$ and $R_2$. Applying the current divider rule results in

$I_{2}^{\prime\prime}=\frac{R_{1}(I)}{R_{1}+R_{2}}=\frac{(12~\Omega)(9~A)}{12~\Omega+6~\Omega}=6~A$

\begin{figure}[h!]
    \centering
    \begin{circuitikz}
        \draw
        (0,0) node[ground]{}
        (0,4) to[R=$R_1$, l=$12\Omega$] (0,0)
        (0,4) -- (4,4)
        (4,4) -- (4,2) to[R=$R_2$, l=$6\Omega$, i=$I_2''$] (4,0)
        (4,4) -- (8,4) to[current source, l=$I=9A$] (8,0)
    \end{circuitikz}
    \caption{Replacing the 36 V voltage source by a short-circuit equivalent to determine the effect of the 9 A current source on current $I_2$.}
    \label{fig:9.11}
\end{figure}

\noindent
Since the contribution to current $I_2$ has the same direction for each source, as shown in Fig. 9.12, the total solution for current $I_2$ is the sum of the currents established by the two sources. That is,

$I_{2}=I_{2}^{\prime}+I_{2}^{\prime\prime}=2A+6~A=8~A$

\begin{figure}[h!]
    \centering
    \begin{circuitikz}
        \draw
        (0,0) node[ground]{}
        (0,3) to[current source, l=$3A$, name=I] (0,0)
        (0,3) -- (3,3) -- (5,3) to[R=$R_1$, l=$6\Omega$] (5,0) -- (0,0);
        (5,3) node[above]{$I_1''$}
    \end{circuitikz}
    \caption{Determining the effect of the 3 A current source on the current $I_1$ in Fig. 9.2.}
    \label{fig:9.4}
\end{figure}

\end{document}

